Human activity detection in outdoor environments is emerging as a very important research field due to its potential application in surveillance, assisted living, search and rescue, and military. On such applications, it is important to have a detailed information about the target, for example, if the detected target is a single person or a group of people, what activity the target is performing, and the rough location of the target. In this paper, we propose novel usage approaches of machine learning techniques to perform human recognition, human activity detection, people counting, and coarse-grained localization by classifying micro-Doppler signatures obtained from a low-cost and low-power radar system. Along with the use of a low-cost, low-power radar system, the experiments presented and evaluated in this paper were performed outdoors with a high-level of clutter, providing important findings that were not previously investigated in the literature on these same conditions. For the feature extraction of the micro-Doppler signatures, a two-directional two-dimensional principal component analysis (2D2PCA) was applied. The results show that the use of this technique greatly improved the classification rate of the Support Vector Machine (SVM) and the \textit{k}-nearest neighbor (\textit{k}NN) classifiers. In addition, we designed and implemented a Convolutional Neural Network (CNN) for the target classifications of type, number, activity, and coarse localization. The classification results obtained by using our CNN model were superior to the ones obtained using the SVM and the \textit{k}NN. This paper also investigates the frame length of the sliding window, the angle of the direction of movement, and the number of radars used in the experiment set up, providing valuable guidelines for machine learning modeling and experimental setup of micro-Doppler based research and applications.