\IEEEPARstart{T}{he} detection, recognition, and classification of human targets and human activities are an increasingly important topic in many applications, such as surveillance, search and rescue, and patient/elderly monitoring. Numerous sensors, such as cameras, LIDAR \cite{madevice2018} and radars, are employed to achieve contactless measurement of humans and human kinetic characteristics. Although a wide variety of sensors exist, Doppler radar is emerging as an increasingly popular device that it is especially useful for motion analysis \cite{narayanan2015radar}. Unlike other sensor technologies, it is all-weather, it works day-and-night, and operates in non-line of sight situations such as through building walls, clothes, and foliage \cite{ram2008simulation}. It is non-intrusive, and it does not generate privacy concerns because the identity or personal identifiable features of a target cannot be obtained with radar detection. Furthermore, low-cost and low-power radar components are becoming more available, which makes Doppler radars more suitable to be deployed outdoors and on a larger scale.

Human activity recognition and classification can be obtained by comparing the differences in the radar micro-Doppler signatures of different targets and activities. A moving target relative to a radar sensor induces a frequency shift of the echo as a result of the well-known Doppler Effect. Additional movements of smaller parts of the target, called micro-motions, will result in additional modulation of the main Doppler frequency shift, known as the \textit{\textbf{micro-Doppler effect}} \cite{balleri2011classification,dura2011human}. The distinctive characteristics of the observed micro-Doppler effect of an object or a process are called \textit{\textbf{micro-Doppler signatures}} \cite{chen2014micro}. For a human activity, a unique micro-Doppler signature is the periodic motion of arms and legs that produce sidebands to the main Doppler frequency \cite{tivive2013image}. Micro-Doppler signatures are typically represented using joint time-frequency analysis such as short-time Fourier transform (STFT) \cite{tivive2013image,chen2000time}.

  In recent years, there has been a great research interest in human activity classification using micro-Doppler signatures \cite{narayanan2015radar,tivive2013image,garcia2014analysis,kim2015human}. In \cite{ccaugliyan2015micro}, a low-power pulse-Doppler radar that operates at 5.8 GHz was used to collect the micro-Doppler signatures of three different activities (walking, running, and crawling) performed by four subjects on a treadmill. Kim and Lin \cite{kim2009human} used a 2.4-GHz Doppler radar to classify seven human activities, including, for example, running, walking, and boxing. I. Bilik et al. \cite{bilik2006gmm} employed a Pulse-Doppler radar operating at 9 GHz to perform automatic target recognition on multiple people, wheeled vehicles, tracked vehicles, and animals. D. P. Fairchild et al.\cite{fairchild2016multistatic} built a bistatic radar system operating at 4 GHz to differentiate three human motions, such as no activity, arm swinging, and picking up an object.
 
In the published works above, various methods, such as Principle Component Analysis (PCA) \cite{mobasseri2009time}, Empirical Mode Decomposition (EMD) \cite{fairchild2014classification}, and Singular Value Decomposition (SVD) \cite{fioranelli2015classification,fioranelli2016performance} were used to extract the micro-Doppler features. These are computer algorithms that extract features automatically and they are more efficient and informative when compared to handcrafted feature extraction, where features are extracted manually by human visual judgement like in \cite{ccaugliyan2015micro,kim2009human,zenaldin2016radar,bjorklund2011millimeter}. 

After being extracted, the micro-Doppler features are fed into classifiers. The most used classifiers in micro-Doppler based human activity detection are Support Vector Machine (SVM) \cite{kim2009human,zenaldin2016radar,zabalza2014robust}, \textit{k}-Nearest-Neighbour (\textit{k}NN) \cite{ccaugliyan2015micro} and Na\"ive Bayes \cite{nanzer2009bayesian}. Deep neural networks have been used to classify micro-Doppler signatures only recently. For example, the authors of \cite{kim2016human} proposed the use of Deep Convolutional Neural Networks (DCNNs) for human detection and activity classification; in \cite{kim2016classification} the authors used DCNNs to classify human swimming styles; and in \cite{tahmoush2010ugs} a DCNN outperformed SVM and \textit{k}NN by a wide margin when classifying multi-target human gait. However, this previous work was implemented indoors or in environments with low levels of clutter. It is more challenging to perform human activity detection outdoors. The complex terrain, the trees, and incident foliage create spatial clutter and introduce noise that results in lower signal-to-noise ratio (SNR). Additionally, animals may be confused with humans, creating false positives. Hence, before performing human activity detection, it is also important to differentiate humans from confusers. Micro-Doppler signatures from animal and vehicle confusers have been investigated, for example in \cite{miller2013micro,lee2017classification,smith2008naive}.

Some research work on human activity recognition outdoors using micro-Doppler signature has considered clutter and noise \cite{zenaldin2016radar,tahmoush2010ugs,karabacak2015knowledge,tahmoush2009radar}. However, the radars used in those works are high--power radars and need constant mains power to function, which is an important shortcoming for lasting continuous detection in outdoors, especially in areas where the electricity supply may be scarce or absent. Our proposed research addresses this problem by using low-power pulsed Doppler radars operating at 5.8 GHz that allows long-lasting battery powered operation. This is an important factor for outdoor operations of surveillance or monitoring.

The outdoors usually comprises a large area, and it is important not only to know what the target is doing but also where the target is.  It is not feasible to perform localization by using range-Doppler analysis in this research, because the pulsed Doppler radar implemented here cannot provide the information of the azimuth and the distance to the target. To the best of our knowledge, there is no relevant literature research applying micro-Doppler signature classification for coarse-grained location estimation. However, our research is investigating ways to estimate approximately the distance between the radar and the target according to the changing intensity of the micro-Doppler signatures due to the atmospheric attenuation, and reflections of the target and the environment. In this research, the radar detection range is split into three non-overlapping ranges, the micro-Doppler signatures are labeled with the correspondent range. Machine learning algorithms such as SVM, \textit{k}NN, and DCNNs are implemented to classify the micro-Doppler signatures based on those ranges.

The main contributions of this paper can be summarized as follows:
\begin{enumerate}
\item The research presented in this paper uses low-cost and low-power Doppler radars to perform micro-Doppler signature based human activity detection in outdoor environments (complex terrains cluttered with plants and animals). This study provides new approaches of usage and results for low-cost radars that have not previously been investigated in the literature.
\item The experiments presented in this paper not only classify human activity but can also differentiate between humans and animals and can count people. For the activity classification, two types of human activities were investigated (walking and running). The differentiation between humans and animals is used to reduce the rate of false alarms resulting from animals. People counting provides the number of individuals performing an activity that could be valuable information in applications pertaining to surveillance, monitoring.
\item An image feature extraction method-2D2PCA has been applied to extract the micro-Doppler features. This paper shows that the use of 2D2PCA improves the performance of SVM and \textit{k}NN greatly. Although CNN can extract features automatically and achieves the best performance, the training process is long and the computation is more costly. By combining 2D2PCA and SVM one can achieve results close to those using CNN, but in less time and with a lower computation cost. The results presented in this paper shows that applying SVM+2D2PCA may provide a more cost-effective and time-efficient option if the application can absorb a small compromise in the performance of the micro-Doppler signature classification.
\item This research investigates a way to roughly estimate the location of human activity by applying micro-Doppler signature classification for coarse-grained location estimation -- this is novel.
\item This research also investigates the effects of frame length of the sliding window, angles of the movement, and the number of radars on the classification performance. Thus it provides valuable guidelines for machine learning modelling optimization and experiment setup for micro-Doppler based research and applications.
\end{enumerate}

The remainder of this paper is organized as follows. Section II introduces the fundamental concepts of micro-Doppler, the characteristics of the Bumblebee radar, and describes the composition of the Doppler radar system in detail. Section III describes the three experiments performed in this research; it presents and analyzes the pattern of spectrograms that were generated by the different subjects, angles, and ranges. Section IV shows how the radar signals were collected and stored in a database, the methods used for data processing and the composition of the samples after data processing. Section V describes the mathematical fundamentals of 2D2PCA and how it can be applied to micro-Doppler feature extraction. Section VI presents the concepts of CNNs and our proposed structure of the CNN for classification of the gathered Micro-Doppler signatures, and models the SVM and \textit{k}NN classifiers by optimizing and listing their hyper-parameters. Section VII presents, analyzes, and compares the classification results of the five classifiers for each of the desired detections (human activity, people counting, human versus animal confuser, and coarse localization). Section VIII investigates the effects of the frame length of the sliding window, angles of the movement, and the number of radars. Section IX summarizes the contributions and discusses the future prospects of the research.