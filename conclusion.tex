This paper applied micro-Doppler signatures to perform four classification tasks, which are human recognition, human activity classification, people counting, and coarse-grained localization. A radar system that consists of two pulsed Doppler radars operating at 5.8 GHz was built. With the collected radar signals and processing them with STFT, the patterns of the spectrograms of different subjects, activities, location ranges, and the number of people were presented and analyzed. Five classifiers, including CNN (RadarNet), SVM, SVM+2D2PCA, \textit{k}NN, \textit{k}NN+2D2PCA were implemented. It was found that the CNN performs the best in all four classification tasks. Also, 2D2PCA was proved to be a very good feature extraction method in micro-Doppler analysis and improved the performance of SVM and \textit{k}NN significantly. At last, three factors, including the frame length of the sliding window, the angle of the movement, and the number of radars were investigated for micro-Doppler signature applications. Our investigation provided a valuable guideline for model optimization and experiment setup of micro-Doppler based research and applications.

In conclusion, this research validates low-power low-cost radars have a great potential for human activity detection, even in outdoor environments with high-levels of clutter. In future work, the method used in this research will be extrapolated into a Doppler radar with a greater detection range and extended to more application scenarios.